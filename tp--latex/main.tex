\documentclass{article}
\usepackage[utf8]{inputenc}
\usepackage{graphicx}
\title{tpLatex}
\author{Ghassen El Mechri}
\date{October 2017}

\begin{document}

\maketitle

\section{Introduction}
\subsection{première sous-section}
\subsection{première sous-sous-section }
\begin{itemize}
\item des poires ;
    \begin{enumerate}
        \item des Williams
        \item des Guyot
    \end{enumerate}
\item des carottes ;
\item des choux.
\end{itemize}

 Le logo de l'école est illustré par la Figure \ref{fig:logo_tncy}


\begin{figure}
    \centering
    \includegraphics[scale=0.1]{logo_TNCY.png}
    \caption{Le logo de TELECOM Nancy}
    \label{fig:logo_tncy}
    \ref{fig:logo_tncy}
\end{figure}

\begin{figure}
   \begin{center}
   \begin{tabular}{|l|c|c|}
   \hline
    colonne 1 & colonne 2 & colonne 3 \\
   \hline
    1.1 & 1.2 & 1.3 \\
    \hline
    2.1 & 2.2 & 2.3 \\
   \hline
   \end{tabular}
  \end{center}
  \caption{Exemple simple de tableau}
  \label{tab:exemple}
 \end{figure}   


Einstein a établi la célèbre formule \ref{eq-Einstein}.
\begin{equation}
  \label{eq-Einstein}
  E = mc^2
\end{equation}


G.R.R Martin a écrit, dans le livre \cite{GOT4}: "“I prefer my history dead. Dead 
history is writ in ink, the living sort in blood.
\bibliographystyle{plain}
\bibliography{biblio.bib}
\end{document}
